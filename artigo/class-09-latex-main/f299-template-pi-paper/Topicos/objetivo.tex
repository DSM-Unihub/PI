a) Desenvolver um algoritmo de PLN capaz de analisar o conteúdo textual de páginas da \textit{Web} e identificar padrões linguísticos associados a conteúdos discriminatórios, injuriosos ou que incitem discurso de ódio.

b) Integrar o sistema de auditoria de conteúdo \textit{Web} com um sistema de notificação, de forma que o usuário seja alertado na tela quando for identificado conteúdo ofensivo ou prejudicial, incluindo informações sobre o motivo do bloqueio e opções para revisão por administradores escolares, caso necessário.

c) Funcionalidade de retroalimentação, permitindo que o sistema aprenda continuamente com novos dados, aprimorando sua capacidade de identificação e bloqueio de conteúdos prejudiciais através do feedback dos administradores ou de atualizações periódicas de dados e padrões linguísticos.

d) Desenvolver um painel de controle administrativo, que permita a visualização de estatísticas sobre tentativas de acesso bloqueadas, tipos de conteúdo identificado, e histórico de notificações, oferecendo também a possibilidade de personalizar critérios de bloqueio de acordo com as políticas da instituição.