% Apresente aqui as conclusões do seu trabalho, verifique se o objetivo foi cumprido, apresenta respostas para o problema da pesquisa, relate as limitações e as recomendações do estudo. Por fim, coloque sugestões para trabalhos futuros.
Este artigo destacou a aplicação da tecnologia e da Inteligência Artificial no contexto escolar, com foco na detecção e no bloqueio de acesso a sites com conteúdos discriminatórios. Diante dos crescentes desafios relacionados à inclusão étnica, educacional e social, a adoção de abordagens inovadoras se torna imprescindível para atender às demandas em constante evolução. 

O projeto Resist está alinhado aos Objetivos de Desenvolvimento Sustentável (ODS) da ONU, especificamente ao objetivo quatro, que visa garantir educação inclusiva, equitativa e de qualidade, e ao objetivo dez, que promove a inclusão social, econômica e política, independentemente de características como idade, sexo, raça, etnia, origem, religião ou condição econômica. A implementação do sistema transcende a simples detecção de conteúdos discriminatórios, abrangendo também o bloqueio de sites que contenham contextos de intolerância. Além disso, possibilita que instituições de ensino identifiquem tentativas de acesso a tais conteúdos, incentivando a conscientização sobre os impactos desses discursos e promovendo uma cultura de respeito e diversidade. Desenvolvido em Python, o sistema extrai o conteúdo de sites, monitora os acessos registrados pelo Squid e retorna informações como URL, data, hora e IP da máquina. 

Atualmente, a aplicação utiliza uma abordagem baseada na presença de palavras específicas para bloquear ou permitir o acesso, sendo limitada pela ausência de integração com a IA já desenvolvida. A URL da página é armazenada em um banco de dados para facilitar a verificação do status de liberação e identificar a palavra ou frase que motivou o bloqueio, se necessário. Após cada verificação, o sistema registra os acessos no banco de dados, permitindo a geração de relatórios detalhados, acessíveis em formato gráfico via Web pelos gestores das instituições. Embora ainda existam limitações, como a possibilidade de bloqueio indevido de conteúdos legítimos de cunho informativo, os resultados obtidos até o momento demonstram que o sistema representa um avanço significativo na utilização da tecnologia para promover valores fundamentais de igualdade, justiça e respeito mútuo. 

O projeto adota um enfoque proativo e holístico, atacando as raízes do racismo e outros discursos de ódio, enquanto inspira esperança e estabelece um modelo para futuras iniciativas de inclusão. Para as melhorias futuras, destacam-se a integração do sistema com a IA treinada e o desenvolvimento de mecanismos de feedbacks contínuo para aprimorar a eficácia e a precisão das análises. Assim, o projeto não apenas se revela economicamente viável, mas também essencial para a construção de uma sociedade mais justa e equitativa, contribuindo diretamente para a promoção de uma educação inclusiva e de qualidade, alinhada aos ODS e aos valores de diversidade e inclusão.