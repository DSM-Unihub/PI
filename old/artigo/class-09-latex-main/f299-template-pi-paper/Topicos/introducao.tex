% A Organização das Nações Unidas (ONU), fundada em 1945, tem como objetivo promover a cooperação internacional entre os países, focando-se nos direitos humanos, desenvolvimento econômico e segurança mundial. Durante as últimas décadas, a ONU expandiu seus horizontes para problemas atuais, impulsionada pela necessidade de abordar assuntos como a desigualdade social e mudanças para garantir uma educação mundial de qualidade, de forma que foi criada a Agenda 2030, que consiste em 17 Objetivos de Desenvolvimento Sustentável (ODS).

% Este artigo busca relacionar-se aos objetivos da organização, aliando-se especificamente a meta quatro, que visa assegurar educação inclusiva e equitativa de qualidade, assim como a meta dez, que promove a inclusão social, econômica e política de todos independentemente de idade, sexo, raça, etnia, origem, religião, condição social e econômica etc. Ambas as ODS traçam um plano universal para alcançar um futuro melhor, especificamente no âmbito social e educacional.

% Na presente análise, compreender a dimensão de conceitos como discurso de ódio e racismo se torna crucial dentro de ambientes web a nível nacional e mundial. Estatísticas revelam, nacional e internacionalmente, a extensão destes problemas e a necessidade de ação imediata para combatê-los. Dentro deste projeto, teremos como objetivo identificar situações de discursos de ódio variados na web, com foco específico no racismo, porém abrangendo outros tipos como sexismo, homofobia, xenofobia etc. de modo generalista.

% Contextualizando a parte do racismo, no País de Gales, que é majoritariamente composto de pessoas brancas, a situação do racismo é recorrente. Em um estudo realizado em 2023, foram entrevistados 170 alunos de diferentes escolas pelo país, onde grande parte delas afirmou já ter presenciado ou sido vítima de situações de preconceito racial dentro das instituições de ensino, com a maioria delas sendo apenas ignorada ou tratada como brincadeira pelos docentes e responsáveis. Um dos grupos de jovens da região centro-leste chegou a relatar ter experienciado uma situação em que alunos de uma escola criaram um grupo no aplicativo de mensagens “WhatsApp” para disseminar frases e xingamentos racistas, além de apologia ao nazismo. Ainda, os alunos da escola que se sentiram prejudicados indagaram que a administração da escola não levou a situação a sério, o que torna a situação ainda mais desgastante. O estudo também promove a normalização do suporte para crianças e adolescentes que experienciam o racismo, promovendo o ensino sobre a diversidade cultural e étnica nas escolas, o diálogo aberto entre alunos e professores, e sanções aos profissionais caso não tratem os casos de discriminação com a devida importância Cifuentes (2023).

% Particularmente relevante para este estudo, ainda no âmbito do racismo, é sua interseção com a educação. A urgência de práticas educacionais antirracistas foi destacada em um relatório recente intitulado “A Urgência da Prática Educacional Antirracista: Educação de Combate às Estatísticas (2023)”, o qual ressalta a necessidade de transformações significativas no sistema educacional para promover a equidade racial e combater manifestações de discriminação dentro e fora das salas de aula. No contexto nacional, estatísticas compiladas pelo Instituto Brasileiro de Geografia e Estatística (IBGE) e outras instituições revelam a persistência do racismo estrutural em diversas esferas da sociedade brasileira. Desde o acesso desigual a serviços básicos até disparidades salariais e representação política, os efeitos do racismo são visíveis e impactam diretamente a vida de milhões de brasileiros. Segundo o Instituto de Referência Negra Peregum (IPEC) em conjunto com o projeto SETA, em pesquisa feita em 127 cidades do país, revelou-se que em média 64\% dos jovens entrevistados apontam o ambiente escolar como o local onde mais sofrem racismo. Vale ressaltar que, na mesma pesquisa, foi apontado que 71\% das pessoas entrevistadas consideram que brancos e negros são tratados diferentemente nestas instituições de ensino. Dito isso, a importância de abordar políticas educacionais afirmativas e inclusivas para negros e outros grupos étnicos se torna indispensável, aliando-se ao combate a desigualdade social no ambiente escolar \textcite{daepistemicidio}.

% Trazendo o contexto para outras formas de discurso de ódio na internet, ao redor do mundo, sua incidência aumenta todos os dias, com aproximadamente 60\% da população usando de redes sociais para se comunicar \textcite{saleh2023detection}.

% A internet onde muitas discussões acontecem o tempo inteiro, e ser anônimo enquanto faz isso acabou por dar liberdade para que muitas pessoas expressassem suas opiniões de forma livre. Contudo, aqueles que discordam destas opiniões de forma “extrema” podem acabar utilizando desta liberdade erroneamente. Compartilhar coisas que gostamos pode acabar se tornando difícil quando existe a ameaça constante de assédio ou comentários tóxicos na web. Esta toxicidade online pode ser chamada de Cyberbullying \textcite{sharma2017detecting}.

% Os desafios no combate ao Cyberbullying podem incluir a detecção da toxicidade extrema quando ocorre e identificar os “predadores” e suas vítimas. Em redes sociais, como o Facebook e o X, por exemplo, onde este está mais presente, incorporam sistemas que automaticamente identificam estas agressões e instâncias de assédio em suas plataformas. Porém, em muitos dos casos, estas ferramentas acabam não sendo tão eficientes quanto deveriam, por falta de refinamento dos métodos de busca e análise do conteúdo \textcite{sharma2017detecting}. 

% Por estes pressupostos, a área de Tecnologia da Informação (TI) emerge como um campo estratégico para desenvolver soluções inovadores que abordem questões complexas como o racismo na internet. Com o constante avanço da Inteligência Artificial (IA), surgem oportunidades promissoras para automatizar processos de auditoria de conteúdo web, especialmente no que diz respeito a identificação e remoção de conteúdos racistas e de discursos de ódio.

% A Inteligência Artificial (IA) representa uma ferramenta poderosa para enfrentar o desafio da auditoria de conteúdo web contendo injúria racial. Por meio de algoritmos sofisticados e técnicas de aprendizado de máquina, a IA pode analisar grandes volumes de dados em tempo real, identificando padrões e contextos associados a discursos racistas e discriminatórios.

% A detecção do conteúdo de ódio não só nas redes sociais, mas em todo o ambiente web (incluindo o escolar) é uma medida necessária e essencial para manter o ambiente propício à paz. Com este projeto, nosso objetivo se torna garantir uma solução com base em experimentação para detectar e bloquear automaticamente estes contextos de racismo e discursos de ódio variados dentro destes ambientes, utilizando de dados coletados em situações reais para alcançar nosso objetivo final.

% O objetivo em questão consiste no desenvolvimento de um sistema de auditoria de conteúdo web em escolas, utilizando-se de Processamento de Linguagem Natural (PLN) para identificar e bloquear sites com contextos de injúria racial. O PLN é uma subárea da IA que se concentra na interação entre computadores e linguagem humana. Por meio de algoritmos e técnicas de análise de texto, o PLN permite que sistemas computacionais compreendam, interpretem e gerem linguagem humana de forma semelhante à dos seres humanos. No contexto deste projeto, o PLN será empregado para analisar o conteúdo textual dos sites acessados por alunos dentro das escolas, identificando padrões linguísticos associados à injúria racial e acionando bloqueios automáticos quando necessário.

% A implementação de um sistema de auditoria de conteúdo web baseado em PLN oferece diversas vantagens para as instituições de ensino. Além de promover um ambiente virtual mais seguro para os alunos, professores e demais membros da comunidade escolar, esta tecnologia contribui para a promoção de uma educação inclusiva e equitativa. Ao bloquear sites com conteúdos injuriosos e racistas, o sistema ajuda a prevenir a propagação de discursos de ódio e contribui para a construção de uma cultura escolar mais consciente e respeitosa. A utilização do PLN permite uma abordagem automatizada, reduzindo a carga de trabalho manual e tornando o processo mais eficiente. Em última análise, a implementação deste sistema reforça o compromisso das escolas com a promoção dos direitos humanos, da igualdade racial e respeito à diversidade, contribuindo para a formação de cidadão conscientes e engajados na construção de uma sociedade justa e inclusiva.

A Organização das Nações Unidas (ONU), fundada em 1945, tem como objetivo promover a cooperação internacional entre os países, focando-se nos direitos humanos, desenvolvimento econômico e segurança mundial. Durante as últimas décadas, a ONU expandiu seus horizontes para problemas atuais, impulsionada pela necessidade de abordar assuntos como a desigualdade social e mudanças para garantir uma educação mundial de qualidade, de forma que foi criada a Agenda 2030, que consiste em 17 Objetivos de Desenvolvimento Sustentável (ODS).

Este artigo busca relacionar-se aos objetivos da organização, aliando-se especificamente a meta quatro, que visa assegurar educação inclusiva e equitativa de qualidade, assim como a meta dez, que promove a inclusão social, econômica e política de todos independentemente de idade, sexo, raça, etnia, origem, religião, condição social e econômica etc. Ambas as ODS traçam um plano universal para alcançar um futuro melhor, especificamente no âmbito social e educacional.

Trazendo o contexto para o discurso de ódio na internet, sua incidência tem aumentado significativamente, acompanhando o crescimento do uso das redes sociais, que são utilizadas por aproximadamente 60\% da população mundial (Ltd, 2020). A internet, como espaço de troca constante de ideias, proporciona liberdade de expressão, mas o anonimato que a acompanha tem sido, em muitos casos, mal utilizado. Essa liberdade tem permitido que indivíduos expressem suas opiniões de forma desmedida, levando, por vezes, a discursos ofensivos ou comportamentos tóxicos. A ameaça constante de assédio ou comentários nocivos na web tornou-se um obstáculo para compartilhar ideias e interesses de forma segura. Essa toxicidade online, que frequentemente resulta em comportamentos de humilhação, perseguição e discriminação, é amplamente conhecida como Cyberbullying.  

Na União Europeia, por exemplo, 80\% das pessoas com acesso à internet relataram ter presenciado discursos de ódio online, enquanto 40\% afirmaram ter se sentido pessoalmente atacadas ou ameaçadas nesses espaços virtuais \cite{castano2021internet}. Já na Coreia do Sul, uma pesquisa realizada pela "National Human Rights Association of Korea" em 2019, envolvendo 1.200 adultos e 500 jovens, revelou dados alarmantes. No grupo de adultos, 64\% relataram ter sido expostos a diferentes formas de discurso de ódio no ano anterior. Entre os principais tipos, destacaram-se ataques relacionados ao local de nascimento (74,6\%), sexismo contra mulheres (68,7\%), discursos contra pessoas idosas (67,8\%), preconceito contra minorias sexuais (67,7\%), imigrantes (66\%) e pessoas com deficiência (58,2\%). No grupo jovem, 68,3\% indicaram ter vivenciado exposições a discursos de ódio, com os casos mais comuns sendo preconceito contra mulheres (63\%) e minorias sexuais (57\%). Importante ressaltar que 83\% das ocorrências foram registradas em redes sociais e outros ambientes digitais \cite{lee2019report}.

No Brasil, o discurso de ódio está amplamente presente tanto nas redes sociais quanto em espaços digitais e físicos, não se limitando ao ambiente online. Em uma análise de 145 notícias publicadas em um único dia (7 de junho de 2016), constatou-se que 90\% delas continham ao menos um comentário de ódio \cite{pelle2016offensive}. Em muitos desses casos, usuários engajados em discussões sobre as notícias acabavam se envolvendo em conflitos, promovendo ataques e ofensas. Outro estudo relevante foi conduzido pela agência Nova/SB, que investigou a intolerância em redes sociais ao longo de três meses, monitorando plataformas como Facebook, Twitter e Instagram. Nesse período, foram registrados 542.781 comentários com teor de discurso de ódio, sendo 84\% deles relacionados a questões negativas, como racismo, misoginia e xenofobia. Esses dados reforçam a necessidade de iniciativas que combatam ativamente o discurso de ódio e promovam a conscientização sobre o impacto dessas práticas nos ambientes digitais e sociais.

Os desafios no combate ao Cyberbullying incluem a identificação ineficiente da toxicidade extrema e a distinção entre "predadores" e vítimas nesse ambiente digital. Redes sociais como Facebook e X (antigo Twitter), onde essas práticas são mais comuns, já incorporam sistemas automatizados para identificar e mitigar agressões e assédios em suas plataformas. Contudo, tais ferramentas frequentemente enfrentam limitações de precisão devido à falta de refinamento nas metodologias de busca e análise do conteúdo \cite{sharma2017detecting}.

Outro ponto relevante, é de que a fase de formação da criança e é de extrema relevância para a construção dos valores da criança em desenvolvimento como também de uma fases de transição do seu eu, e de suas relação sociais. Assim como, as escolas podem criar um ambiente que venha a constituir-se num "espelho" e num "mundo" para as crianças, ajudando-as a caminhar para fora de um ambiente familiar adverso \cite{szymanski2007relacao}.

Portanto, estes dados e desafios evidenciam a necessidade de soluções mais avançadas, que consigam detectar nuances e contextos complexos de discurso de ódio na web, abrangendo categorias como sexismo, homofobia, xenofobia, racismo etc., tornando possível a promoção de um ambiente digital mais seguro e inclusivo, principalmente em instituições de ensino, para que dessa forma, elas sejam capazes de conscientizar alunos que acessam ou apresentam falas de cunho preconceituoso. 

Diante disso, a área de Tecnologia da Informação (TI) se destaca como um campo estratégico para o desenvolvimento de soluções inovadoras capazes de abordar questões complexas, como o discurso de ódio na internet. Com os constantes avanços na Inteligência Artificial (IA), surgem oportunidades promissoras para automatizar processos de auditoria de conteúdo web, especialmente no que diz respeito à identificação e mitigação de discursos de ódio. A IA desponta como uma ferramenta poderosa nesse cenário, oferecendo a capacidade de analisar grandes volumes de dados em tempo real por meio de algoritmos avançados e técnicas de aprendizado de máquina. A detecção de conteúdo injurioso, não apenas nas redes sociais, mas em todo o ambiente web — particularmente no contexto escolar —, é uma medida essencial para garantir um ambiente de convivência pacífica. 

Este projeto, portanto, desenvolve uma solução que permite identificar e bloquear automaticamente esses contextos acessados no ambiente escolar, utilizando dados coletados em situações reais para atingir resultados robustos e confiáveis, assim como empregando o Processamento de Linguagem Natural (PLN) para identificar e bloquear sites que apresentem contextos de discurso de ódio. 

O PLN, uma subárea da IA, concentra-se na interação entre computadores e a linguagem humana. Ele utiliza algoritmos e técnicas de análise textual que permitem aos sistemas computacionais compreender, interpretar e gerar linguagem humana de forma semelhante aos seres humanos. No contexto deste projeto, o PLN será empregado para analisar o conteúdo textual de sites acessados por alunos, identificando padrões linguísticos associados a discursos de ódio. Todos os dados utilizados para o treinamento da IA foram ajustados por meio do modelo pré-treinado Bidirectional Encoder Representations from Transformers (BERT).  

O BERT, desenvolvido pelo Google, é um modelo de aprendizagem profunda projetado especificamente para tarefas de Processamento de Linguagem Natural. Ele é reconhecido por sua capacidade de compreender o contexto completo das palavras dentro de uma frase, graças ao seu treinamento bidirecional. Diferentemente de outros modelos, como o Word2Vec ou o GloVe, o BERT analisa as palavras considerando todo o contexto da sentença, o que o torna mais eficaz na identificação de nuances linguísticas complexas \cite{saleh2023detection}.

A implementação de um sistema de auditoria de conteúdo web baseado em PLN oferece inúmeros benefícios para instituições de ensino. A automação proporcionada por todo o sistema reduz significativamente o esforço manual necessário para auditorias, tornando o processo mais eficiente e escalável. Em última análise, a adoção desse sistema reafirma o compromisso das escolas com a promoção dos direitos humanos, igualdade e respeito à diversidade, formando cidadãos mais conscientes e engajados na construção de uma sociedade justa e inclusiva.