%%%% fatec-article.tex, 2024/03/10

%% Classe de documento
\documentclass[
landscape,
  a4paper,%% Tamanho de papel: a4paper, letterpaper (^), etc.
  12pt,%% Tamanho de fonte: 10pt (^), 11pt, 12pt, etc.
  english,%% Idioma secundário (penúltimo) (>)
  brazilian,%% Idioma primário (último) (>)
]{article}

%% Pacotes utilizados
\usepackage[]{fatec-article}
\usepackage{setspace}

%% Processamento de entradas (itens) do índice remissivo (makeindex)
%\makeindex%

%% Arquivo(s) de referências
%\addbibresource{fatec-article.bib}

%% Início do documento
\begin{document}

% Seções e subseções
%\section{Título de Seção Primária}%

%\subsection{Título de Seção Secundária}%

%\subsubsection{Título de Seção Terciária}%

%\paragraph{Título de seção quaternária}%

%\subparagraph{Título de seção quinária}%

%\section*{Diário de Bordo}%
% \section*{Instruções para o preenchimento}
% \doublespacing
% \begin{enumerate}
%     \item O Diário de Bordo é usado para registrar atividades, progressos, ideias e desafios enfrentados em um projeto ou durante a rotina de trabalho. Serve como um registro cronológico e detalhado das operações diárias, facilitando a organização e o acompanhamento das tarefas.
%     \doublespacing
%     \item Durante o registro das atividades deve-se incluir detalhes como datas, horários, descrições de tarefas, nomes de participantes e observações relevantes.  Esta documentação contínua ajuda na avaliação do progresso de projetos ou atividades, permitindo ajustes e melhorias contínuas nos processos.
%     \doublespacing
%     \item Para evidenciar a realização das tarefas, você poderá utilizar a criação de anexos para adicionar anotações, fotos, prints, questionários, entre outros.
% \end{enumerate}

 \break

 \begin{table}[]
\centering
\begin{tabular}{|l|l|l|l|l|}
\hline
Nome da Atividade & Data de início & Data de término & Responsável pela atividade & Descrição da atividade realizada \\ \hline
Metodologia                 &  01/08/24  & 15/08/24  & Isabela Chaves        & Desenvolver metodologia                    \\\hline
Mudança de Título           &  16/08/24  & 30/08/24  & Ruth Mendonça         & Alterar título do projeto                  \\\hline
Site Web Responsivo         &  01/09/24  & 15/09/24  & Daniel Mandira        & Fazer site responsivo com Next.js          \\\hline
Fluxograma Metodologia      &  16/09/24  & 30/09/24  & Isabela Chaves        & Fazer fluxograma da metodologia            \\\hline
Estado da Arte              &  01/10/24  & 15/10/24  & Caio Moraes           & Pesquisar artigos correlatos               \\\hline
Introdução Refeita          &  16/10/24  & 30/10/24  & Caio Moraes           & Refazer a introdução                        \\\hline
Treinamento IA              &  01/11/24  & 08/11/24  & Ruth Mendonça         & Treinar a IA                                \\\hline
Integrar IA ao Sistema      &  09/11/24  & 15/11/24  & Bruno Freitas         & Conectar IA ao sistema                     \\\hline
Integração Notificação      &  01/09/24  & 15/09/24  & Bruno \& Daniel       & Integrar sistema de notificação            \\\hline
Logbook                     &  16/09/24  & 30/09/24  & Isabela Chaves        & Atualizar o logbook do projeto             \\\hline
Diagrama Não Relacional     &  01/10/24  & 15/10/24  & Isabela Chaves        & Criar diagrama não relacional              \\\hline
Mudar para Python           &  16/10/24  & 30/10/24  & Daniel Mandira        & Migrar código para Python                  \\\hline
Refazer Banner              &  01/11/24  & 08/11/24  & Caio Moraes           & Refazer o banner do projeto                \\\hline
Servidor - URL Completa     &  09/11/24  & 15/11/24  & Mauricio Bertoldo     & Configurar servidor                        \\\hline
API MongoDB                 &  01/10/24  & 15/10/24  & Daniel Mandira        & Criar API com MongoDB                      \\\hline



  
\end{tabular}
\end{table}



\end{document}